\chapter{Links do Projeto}

Apresentação dos links e QRcodes necessários para acessar na internet itens relacionados ao projeto diversaGente.  


\begin{itemize}
	\item Link do Youtube
\end{itemize}
\begin{figure}[htb]
	\includegraphics[width=0.40\textwidth]{anexos/youtube.png} \\
	\hyperlink {Link Youtube}{https://www.youtube.com/channel/UCs1XOD3RrJ2OcNL3p-R0kBQ}
\end{figure}

\newline

\begin{itemize}
	\item Link do Blog,
\end{itemize}
\begin{figure}[htb]
	\includegraphics[width=0.40\textwidth]{anexos/blog.png} \\
	\hyperlink {Link do Blog}{https://garagelaunch.blogspot.com/}
\end{figure}

\pagebreak

\begin{itemize}
	\item Link do Git
\end{itemize}
\begin{figure}[htb]
	\includegraphics[width=0.40\textwidth]{anexos/blog.png} \\
	\hyperlink {Link do Blog}{https://garagelaunch.blogspot.com/}
\end{figure}

\newline

\begin{itemize}
	\item Link do SVN
\end{itemize}
\begin{figure}[htb]
	\includegraphics[width=0.40\textwidth]{anexos/svn.png} \\
	\hyperlink {Link do Blog}{https://https://svn.spo.ifsp.edu.br/svn/a6pgp/}
\end{figure}

\pagebreak

\begin{itemize}
	\item Link da Prova Conceito
\end{itemize}
\begin{figure}[htb]
	\includegraphics[width=0.40\textwidth]{anexos/heroku.png} \\
	\hyperlink {Link do Blog}{https://https://dev-diversagente.herokuapp.com}
\end{figure}

\newline

%---------------------------------------------------------------

% Please add the following required packages to your document preamble:
% \usepackage{lscape}
% \usepackage{longtable}
% Note: It may be necessary to compile the document several times to get a multi-page table to line up properly
\renewcommand\LTcaptype{quadro}
\begin{quadro}
	\centering
	\ABNTEXfontereduzida
	\caption[Custo das ferramentas]{Custo das ferramentas}
	\label{quadro-exemplo}
	\begin{longtable}[]{|l|c|c|c|}
		\hline
		& Prova Conceito  &  MVP  & Projeto Finalizado   \\ \hline
		\endfirsthead
		%
		\multicolumn{4}{c}{\scriptsize Fonte: Equipe diversaGente (2022).}%
		{{\bfseries Quadro \thetable\ continued from previous page}} \\
		\endhead
		%
		Ambientalização & X & X & X \\ \hline
		Integração de ferramentas & X & X & X \\ \hline
		Fórum &  & X & X \\ \hline
		Chat  &  &  & X \\ \hline
		Avaliações &  & X & X \\ \hline
		Feed de notícias &  &  & X \\ \hline
				Ambientalização & X & X & X \\ \hline
	\end{longtable}
\fonte{Equipe diversaGente (2022)}
\end{quadro}



%-------------------------------------------------------------

\begin{quadro}[htb]
	\centering
	\ABNTEXfontereduzida
	\caption[Quadro Teste ]{Quadro Teste}
	\label{item}
\end{quadro}

\begin{landscape}
	\begin{longtable}{|p{3.3cm}|p{10.3cm}|}
		\hline
		\thead{} & \thead{Ator} \\
		\hline
				\endfirsthead
		%
		\multicolumn{2}{c}{\scriptsize Fonte: Equipe diversaGente (2022).}%
		{{\bfseries Quadro \autoref{item}\ continued from previous page}} \\
		\endhead
		Descrição & Ao criar uma subcategoria, o cliente poderá excluir sua própria mensagem dentro do canal de texto e poderá excluir as mensagens de outros dentro dessa subcategoria que ele criou. O Administrador terá permissão de excluir as mensagens das subcategorias, até mesmo do criador dela.\\
		\hline
		Fluxo Básico  & 
		Para o cliente:
		\begin{enumerate}
			\item O usuário seleciona a seção "Fórum";
			\item O usuário seleciona a categoria desejada;
			\item O sistema mostra todas as subcategorias relacionadas à categoria escolhida pelo usuário;
			\item O usuário escolhe uma subcategoria;
			\item O sistema mostra todas as mensagens ocorridas nesse canal de texto até o momento. 
			\item O usuário seleciona a mensagem que deseja excluir;
			\item O sistema atualiza o canal de texto retirando a mensagem excluída. 
			\item O  caso de uso é encerrado. 
		\end{enumerate}\\
		\hline
		Fluxo Básico  & 
		Para o Administrador:
		\begin{enumerate}
			\item O administrador seleciona a seção "Fórum";
			\item O administrador seleciona a categoria desejada;
			\item O administrador mostra todas as subcategorias relacionadas à categoria escolhida pelo usuário;
			\item O administrador escolhe uma subcategoria;
			\item O sistema mostra todas as mensagens ocorridas nesse canal de texto até o momento. 
			\item O administrador seleciona a mensagem que deseja excluir;
			\item O sistema atualiza o canal de texto retirando a mensagem excluída. 
			\item O caso de uso é encerrado. 
		\end{enumerate}\\
		\hline
		Pré-condições & Para o Cliente: 
		
		O cliente deve estar logado no app e ser o criador do subcategoria escolhida.\\
		\hline
		Pré-Condições & Para o Administrador:
		
		Deve-ser ter o permissionamento de acesso necessário.\\
		\hline
		Pós-condições & O sistema deve atualizar o canal de texto já retirando a mensagem de texto excluída.\\
		\hline
		\thead{} & \thead{Ator} \\
		\hline
		Descrição & Ao criar uma subcategoria, o cliente poderá excluir sua própria mensagem dentro do canal de texto e poderá excluir as mensagens de outros dentro dessa subcategoria que ele criou. O Administrador terá permissão de excluir as mensagens das subcategorias, até mesmo do criador dela.\\
		\hline
		Fluxo Básico  & 
		Para o cliente:
		\begin{enumerate}
			\item O usuário seleciona a seção "Fórum";
			\item O usuário seleciona a categoria desejada;
			\item O sistema mostra todas as subcategorias relacionadas à categoria escolhida pelo usuário;
			\item O usuário escolhe uma subcategoria;
			\item O sistema mostra todas as mensagens ocorridas nesse canal de texto até o momento. 
			\item O usuário seleciona a mensagem que deseja excluir;
			\item O sistema atualiza o canal de texto retirando a mensagem excluída. 
			\item O  caso de uso é encerrado. 
		\end{enumerate}\\
		\hline
		Fluxo Básico  & 
		Para o Administrador:
		\begin{enumerate}
			\item O administrador seleciona a seção "Fórum";
			\item O administrador seleciona a categoria desejada;
			\item O administrador mostra todas as subcategorias relacionadas à categoria escolhida pelo usuário;
			\item O administrador escolhe uma subcategoria;
			\item O sistema mostra todas as mensagens ocorridas nesse canal de texto até o momento. 
			\item O administrador seleciona a mensagem que deseja excluir;
			\item O sistema atualiza o canal de texto retirando a mensagem excluída. 
			\item O caso de uso é encerrado. 
		\end{enumerate}\\
		\hline
		Pré-condições & Para o Cliente: 
		
		O cliente deve estar logado no app e ser o criador do subcategoria escolhida.\\
		\hline
		Pré-Condições & Para o Administrador:
		
		Deve-ser ter o permissionamento de acesso necessário.\\
		\hline
		Pós-condições & O sistema deve atualizar o canal de texto já retirando a mensagem de texto excluída.\\
		\hline
	\end{longtable}
\end{landscape}
