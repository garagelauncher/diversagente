% ---
% RESUMOS
% ---

% resumo em português
\setlength{\absparsep}{18pt} % ajusta o espaçamento dos parágrafos do resumo
\begin{resumo}

O crescimento exponencial do uso da internet na última década, levaram as redes sociais, que são pautadas em diversos assuntos, a mostrarem o maior centro de socialização da sociedade moderna. Percebendo que, em meio a essa infinidade, existe a necessidade de evidenciação de algumas dessas pautas por vezes negligenciadas frente aos tópicos dominantes nas redes, surge o diversaGente. Esse é um aplicativo de rede social focado na melhoria de vida de pessoas com neurodiversidades e suas famílias a partir da criação de um ambiente em que elas sintam-se confortáveis para trocar experiências. E para tornar essa aplicação possível, a metodologia ágil foi seguida para organizar e desenvolver do projeto baseiando-se no Scrum. As tecnologias utilizadas no desenvolvimento integram o ecossistema JavaScript, as quais incluem React Native, Typescript, Node.JS, Nest.JS e Prisma.

 \textbf{Palavras-chaves}: rede social. neurodiversidades. projeto.

\end{resumo}

% resumo em inglês
\begin{resumo}[Abstract]
 \begin{otherlanguage*}{english}

The exponential growth of internet use in the last decade, led social networks, which are based on various subjects, to show the greatest socialization center of modern society. Realizing that, in the midst of this infinity, there is a need to highlight some of these guidelines that are sometimes neglected in the face of the dominant topics on the networks, the diverseGente appears. This is a social network application focused on improving the lives of people with neurodiversities and their families by creating an environment in which they feel comfortable to exchange experiences. And to make this application possible, the agile methodology was followed to organize and develop the project based on Scrum. The technologies used in development are part of the JavaScript ecosystem, which include React Native, Typescript, Node.JS, Nest.JS and Prisma.

   \textbf{Keywords}: social networks. neurodiversities. project.
 \end{otherlanguage*}
\end{resumo}