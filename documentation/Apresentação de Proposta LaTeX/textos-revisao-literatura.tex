% ---
% Capitulo de revisão de literatura
% ---
\chapter{Revisão da Literatura}

Neste tópico será feito um estudo e levantadas observações mais a fundo a respeito da neurodiversidade, relatar quando começaram os estudos a respeito das pessoas neurodiversas e entender qual foi/é o papel da sociedade, no que isso acarretou e as mudanças ocorridas e necessárias com enfoque neste público.

\section{Influência do termo}
As questões neurodiversas em toda sociedade sempre foi um assunto difícil de falar, principalmente porque no passado era tratado com "modelo de tragédia pessoal" \cite{oliver1990politics}. Dessa maneira, foi crescendo o indiferente e inquestionável a respeito das pessoas que sofrem algum tipo de neurodiversidade.Entretanto, com os estudos, os teóricos que estudavam a causa mais fundo diziam que esse modelo social estaria equivocado e como diz o Dr. Lawrence Fung em seu projeto de estudos \cite{lawrencefung}, o conceito de neurodiversidade é apenas um conceito de descrever melhor os nossos diferentes cérebros. 

Esse novo conceito que foi surgindo ao longo do tempo, sofreu e sofre até hoje dificuldades para as pessoas entenderem e conviverem, uma vez que se manifestam no convívio social. Essa dificuldade também pode ser explicada por ser um movimento recente, que surgiu no final da década de 90 pela socióloga Judy Singer durante seus trabalhos científicos na University of Technology Sydney. Nele, ela afirma que a prática de conversar sobre a neurodiversidade passa das fronteiras que antes eram só visualizadas pelos pais e/ou por pessoas que convivam com alguma diferença. Somado a isso, a autora também explica que a composição do termo Neurodiversidade se aplica com a junção de tudo que acontece com o sistema nervoso acrescido à maneiras diferentes que ele se comporta nos seres humanos. 

Somado a isso, podemos entender que os estudiosos, médicos e sociólogos batalham para defender e mostrar que quando se fala de composições neurológicas diferentes é basear que todos os seres humanos têm mentes diferentes, ou seja, não existe uma mente igual ou melhor. "Não há duas mentes neste mundo que possam ser exatamente iguais." \cite{judysinger}. Dessa maneira, podemos encaixar cada indivíduo como sendo únicos e importantes e empatizar com as diferenças do outro. 


\section{Virada de responsabilidades}
Um dos motivos que nos levou a trazer o tema da neurodiversidade para o projeto final está na maneira que o assunto é tratado na sociedade. Há muitos tabus a serem derrubados. Se, por volta de 1960, ainda se culpavam os pais pelos seus filhos terem nascido com alguma neurodiversidade e, portanto, impossibilitava o surgimentos de movimentos que pudessem entender e ajudar os familiares, hoje temos algumas conquistas a serem celebradas como projetos de leis que protejam as pessoas neurodiversas graças aos estudos sobre deficiência nos anos 70 que desconstruiu um modelo de responsabilidade única para uma socialmente construída.\cite{ortega2008}. 

Os benefícios para a inclusão de todos em sociedade traz a responsabilidade que moldam todos os aspectos que vivemos, ou seja, não apenas o Estado e as famílias que passam a entender e abrir espaço para todos, mas o meio privado passa a acolher práticas para incluir, criar e ampliar programas de inclusão em seus meios produtivos. Contudo, é necessário um ponto de atenção para não se criar rótulos, pois como já foi citado pela sociologia Singer, "Nós somos todos habitantes neuro diversos do planeta" \cite{judysinger}. Ou seja, habilidades que podem ser facilmente realizadas por pessoas que têm TEA, por exemplo, quem tem TDAH pode ter enorme dificuldade para executar a mesma tarefa. Isso, se aplica, logicamente, para quem não tem nenhum dos dois e podem ter a mesma dificuldade ou facilidade. Assim, é necessário saber como interagir com as diferentes pessoas, tratando todos como profissionais, sem diferenciá-los.

Assim, a virada de responsabilidade se junta desde quando começou a ser abordado na década de 60, passando por estudos, diagnósticos, tratamentos, publicações e políticas públicas e privadas de acolhimento e entendimento da causa. Isso é fundamental para que uma sociedade conviva cada vez mais unida e responsável sobre as diferenças. 


\section{Responsabilidade do aplicativo}
Evidencia-se que no Brasil há um embate entre profissionais e pais a respeito do movimento neurodiversos \cite{rios}, uma vez que a sociedade brasileira ainda está atrelada à modelos ultrapassados da década de 1960. Ou seja, fora todas as barreiras encontradas na sociedade mundialmente, evidentemente na população brasileira isso não seria diferente.

Dessa maneira, o diversaGente tem como objetivo tornar-se um ambiente de discussões e compartilhamento de informações sobre o assunto. Assim, o comprometimento na atual era digital frente ao diálogo sobre neurodiversidades nos remete ao compartilhamento de mais informações a respeito do assunto com pessoas em diversos lugares, diversas famílias e culturas, pois em diversos artigos apresentam críticas a forma que os métodos atuais de tratamento são usados, uma vez que muitos tratamentos centram-se na deficiência e não na forma humana que deve ser tratado\cite{machado}. 

A responsabilidade da nossa aplicação, portanto, está no auxílio na sociedade como um todo e não apenas em um nicho específico, como médicos ou especialistas. Assim, podemos representar em prática aquilo que a socióloga Singuer já afirma a respeito do significado da palavra neurodiversa e como deve ser tratada em sociedade, uma vez que nosso desenvolvimento se aplicou em atender a todos. 



% ---
